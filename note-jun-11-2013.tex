%\documentclass{article}
\documentclass[12pt]{article}
\usepackage{latexsym}
\usepackage{amsmath}
\usepackage{amssymb}
\usepackage{relsize}
\usepackage{geometry}
\geometry{letterpaper}

%\usepackage{showlabels}

\textwidth = 6.0 in
\textheight = 8.5 in
\oddsidemargin = 0.0 in
\evensidemargin = 0.0 in
\topmargin = 0.2 in
\headheight = 0.0 in
\headsep = 0.0 in
%\parskip = 0.05in
\parindent = 0.35in


%% common definitions
\def\stackunder#1#2{\mathrel{\mathop{#2}\limits_{#1}}}
\def\beqn{\begin{eqnarray}}
\def\eeqn{\end{eqnarray}}
\def\nn{\nonumber}
\def\baselinestretch{1.1}
\def\beq{\begin{equation}}
\def\eeq{\end{equation}}
\def\ba{\beq\new\begin{array}{c}}
\def\ea{\end{array}\eeq}
\def\be{\ba}
\def\ee{\ea}
\def\stackreb#1#2{\mathrel{\mathop{#2}\limits_{#1}}}
\newcommand{\gsim}{\lower.7ex\hbox{$
\;\stackrel{\textstyle>}{\sim}\;$}}
\newcommand{\lsim}{\lower.7ex\hbox{$
\;\stackrel{\textstyle<}{\sim}\;$}}
\newcommand{\nfour}{${\mathcal N}=4$ }
\newcommand{\ntwo}{${\mathcal N}=2$ }
\newcommand{\ntwon}{${\mathcal N}=2$}
\newcommand{\ntwot}{${\mathcal N}= \left(2,2\right) $ }
\newcommand{\ntwoo}{${\mathcal N}= \left(0,2\right) $ }
\newcommand{\none}{${\mathcal N}=1$ }
\newcommand{\nonen}{${\mathcal N}=1$}
\newcommand{\vp}{\varphi}
\newcommand{\pt}{\partial}
\newcommand{\ve}{\varepsilon}
\newcommand{\gs}{g^{2}}
\newcommand{\qt}{\tilde q}

%%
\newcommand{\p}{\partial}
\newcommand{\wt}{\widetilde}
\newcommand{\ov}{\overline}
\newcommand{\mc}[1]{\mathcal{#1}}
\newcommand{\md}{\mathcal{D}}

\newcommand{\GeV}{{\rm GeV}}
\newcommand{\eV}{{\rm eV}}
\newcommand{\Heff}{{\mathcal{H}_{\rm eff}}}
\newcommand{\Leff}{{\mathcal{L}_{\rm eff}}}
\newcommand{\el}{{\rm EM}}
\newcommand{\uflavor}{\mathbf{1}_{\rm flavor}}
\newcommand{\lgr}{\left\lgroup}
\newcommand{\rgr}{\right\rgroup}

\newcommand{\Mpl}{M_{\rm Pl}}
\newcommand{\suc}{{{\rm SU}_{\rm C}(3)}}
\newcommand{\sul}{{{\rm SU}_{\rm L}(2)}}
\newcommand{\sutw}{{\rm SU}(2)}
\newcommand{\suth}{{\rm SU}(3)}
\newcommand{\ue}{{\rm U}(1)}
%%%%%%%%%%%%%%%%%%%%%%%%%%%%%%%%%%%%%%%
%  Slash character...
\def\slashed#1{\setbox0=\hbox{$#1$}             % set a box for #1
   \dimen0=\wd0                                 % and get its size
   \setbox1=\hbox{/} \dimen1=\wd1               % get size of /
   \ifdim\dimen0>\dimen1                        % #1 is bigger
      \rlap{\hbox to \dimen0{\hfil/\hfil}}      % so center / in box
      #1                                        % and print #1
   \else                                        % / is bigger
      \rlap{\hbox to \dimen1{\hfil$#1$\hfil}}   % so center #1
      /                                         % and print /
   \fi}                                        %

%%EXAMPLE:  $\slashed{E}$ or $\slashed{E}_{t}$

%%

\newcommand{\LN}{\Lambda_\text{SU($N$)}}
\newcommand{\sunu}{{\rm SU($N$) $\times$ U(1)} }
\newcommand{\sunun}{{\rm SU($N$) $\times$ U(1)}}
\def\cfl {$\text{SU($N$)}_{\rm C+F}$ }
\newcommand{\mUp}{m_{\rm U(1)}^{+}}
\newcommand{\mUm}{m_{\rm U(1)}^{-}}
\newcommand{\mNp}{m_\text{SU($N$)}^{+}}
\newcommand{\mNm}{m_\text{SU($N$)}^{-}}
\newcommand{\AU}{\mc{A}^{\rm U(1)}}
\newcommand{\AN}{\mc{A}^\text{SU($N$)}}
\newcommand{\aU}{a^{\rm U(1)}}
\newcommand{\aN}{a^\text{SU($N$)}}
\newcommand{\baU}{\ov{a}{}^{\rm U(1)}}
\newcommand{\baN}{\ov{a}{}^\text{SU($N$)}}
\newcommand{\lU}{\lambda^{\rm U(1)}}
\newcommand{\lN}{\lambda^\text{SU($N$)}}
%\newcommand{\Tr}{{\rm Tr\,}}
\newcommand{\bxir}{\ov{\xi}{}_R}
\newcommand{\bxil}{\ov{\xi}{}_L}
\newcommand{\xir}{\xi_R}
\newcommand{\xil}{\xi_L}
\newcommand{\bzl}{\ov{\zeta}{}_L}
\newcommand{\bzr}{\ov{\zeta}{}_R}
\newcommand{\zr}{\zeta_R}
\newcommand{\zl}{\zeta_L}
\newcommand{\nbar}{\ov{n}}

\newcommand{\loU}{\lambda_0^{\rm U(1)}}
\newcommand{\llU}{\lambda_1^{\rm U(1)}}
\newcommand{\loN}{\lambda_0^\text{SU($N$)}}
\newcommand{\llN}{\lambda_1^\text{SU($N$)}}
\newcommand{\poU}{\psi_0^{\rm U(1)}}
\newcommand{\plU}{\psi_1^{\rm U(1)}}
\newcommand{\poN}{\psi_0^\text{SU($N$)}}
\newcommand{\plN}{\psi_1^\text{SU($N$)}}

\newcommand{\CPC}{CP($N-1$)$\times$C }
\newcommand{\CPCn}{CP($N-1$)$\times$C}

\newcommand{\MN}{M_\text{SU($N$)}}
\newcommand{\MU}{M_{\rm U(1)}}

\newcommand{\Tr}{\text{Tr}}
\newcommand{\Ts}{\text{Ts}}
\newcommand{\dm}{\hat{{\scriptstyle \Delta} m}}

\begin{document}

	To remind, the following $ F $-terms contribute to the part of the worldsheet potential linear in $ \mu $:
\beq
\label{Fterms}
	g_2^2~ \Tr\, \Big|\, \Ts\, q \wt{q} ~+~ \sqrt{2}\, \mu_2 \cdot \aN \,\Big|^2  ~~+~~
	\frac{1}{2}\,g_1^2~ \Big|\, \Tr\, q\wt{q} ~+~ \sqrt{N}\, \mu_1 \cdot a \Big|^2\,.
\eeq
	Here $ q ~=~ q^{kA} $ is the quark matrix, $ \aN $ is the non-abelian adjoint (matrix), and $ a $ is the U(1) ``adjoint'',
	``$ \Ts $'' stands for the traceless part of a matrix.
	These are just the usual $ F $-terms written in the matrix form, nothing else.

	Extracting the linear terms in $ \mu $, we obtain the contributions to the worldsheet potential
\begin{align}
%
\notag
	V  ~~\supset~~ & -~~ 2\pi\, \int N r\, dr\,\, \frac{1}{2}\, g_1^2\, \sqrt{\frac{2}{N}}\, \Big( \phi_1^2 \,+\, (N-1)\phi_2^2 \Big)
						\lgr \mu_1\, m  ~+~ \ov{\mu}{}_1\, \ov{m} \rgr  ~~-~~
	\\
%
	& 
	-~~ 2\pi\, \int r\, dr\, g_2^2\, \Big( \phi_1^2 \,-\, \phi_2^2 \Big) 
						\lgr \mu_2\, (\ov{n}\, \dm\, n)  ~+~  \ov{\mu}{}_2\, ( \ov{n}\, \dm{}^\dag\, n ) \rgr.
\end{align}
	Here $ m $ is the average mass, and $ \dm $ is the mass difference matrix.
	The first term seemingly is not a potential at all but just a constant --- but please read further.

	The profile integrals here are integrable, as they are really the right-hand sides of the first order equations 
	for the gauge functions $ f(r) $ and $ f_N(r) $. 
	So the integrands are just the full derivatives of these functions and the integrals are trivially taken.
	This \emph{would be} alright, but the first integral 
\beq
	\int N r\, dr\,\, \frac{1}{2}\, g_1^2\, \Big( \phi_1^2 \,+\, (N-1)\phi_2^2 \Big)
\eeq
	is missing a $ N\xi $ in the bracket.
	So this is ``almost'' a full-derivative of $ f(r) $.
	The difference will be 
\[
	\int\, N \xi\,\, r\, dr
\]
	integrated over the whole transverse plane.
	It looks as though we hit one of those infrared divergencies. 

	Indeed, when calculating the cross-terms ({\it i.e.} terms which are linear in $ \mu $), we used the zero-order (in $ \dm $) 
	quark profile matrix $ q $.
	This is justified in the first term in Eq.\,\eqref{Fterms}, since $ \aN $ is already proportional to $ \dm $,
	but this is \emph{not} justified in the second term, since, as we know, $ a $ just equals its vacuum value, which is just the 
	average mass $ m $.

\pagebreak
	{\bf (1)}\;\;  So, perhaps, the right way to calculate the second term in Eq.\,\eqref{Fterms} is to include the linear in $ \dm $ corrections
	to the quark profiles. 
	It maybe that, instead of introducing $ N $ quark profile functions, we could just use
\[
	\varphi ~~=~~ \phi_2  ~+~  n\nbar\, (\phi_1 \,-\, \phi_2)  ~+~  \dm \cdot \phi_m  ~+~ \dots
\]
	with a single new function $ \phi_m $ --- whereas $ \phi_1 $ and $ \phi_2 $ are the usual functions approaching $ \sqrt{\xi} $
	(meaning, the average $ \xi $).
	Preliminarily, this does seem to do the trick, but the first order equation for $ \phi_m $ needs
	to be derived before anything can be calculated


\vspace{0.8cm}


	{\bf (2)}\;\;  It seems as though we should also include linear corrections into the singlet field ``$ a $'' as well.
	On the second thought, if $ a $ (the abelian field!) had corrections linear in $ \dm $, {\it e.g.}
$ (\nbar\, \dm\, n) $,
	we would not have the correct
	supersymmetric limit ($ \mu \,=\, 0 $). As we know, in the supersymmetric case only $ \aN $ develops a profile, while
	$ a $ stays completely condensed. Of course, this ``supersymmetric'' case is not really a true $ \mu \,\longrightarrow\, 0 $ limit,
	but also requires an introduction of the FI $ D $-term (since the effective FI $ F $-term is gone when $ \mu $ goes to zero).
	But it \emph{may be} that corrections to $ a $ are in fact proportional to $ \mu $ instead, and so we can drop them
	for the current purpose

\end{document}
