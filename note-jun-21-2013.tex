%\documentclass{article}
\documentclass[12pt]{article}
\usepackage{latexsym}
\usepackage{amsmath}
\usepackage{amssymb}
\usepackage{relsize}
\usepackage{geometry}
\geometry{letterpaper}

%\usepackage{showlabels}

\textwidth = 6.0 in
\textheight = 8.5 in
\oddsidemargin = 0.0 in
\evensidemargin = 0.0 in
\topmargin = 0.2 in
\headheight = 0.0 in
\headsep = 0.0 in
%\parskip = 0.05in
\parindent = 0.35in


%% common definitions
\def\stackunder#1#2{\mathrel{\mathop{#2}\limits_{#1}}}
\def\beqn{\begin{eqnarray}}
\def\eeqn{\end{eqnarray}}
\def\nn{\nonumber}
\def\baselinestretch{1.1}
\def\beq{\begin{equation}}
\def\eeq{\end{equation}}
\def\ba{\beq\new\begin{array}{c}}
\def\ea{\end{array}\eeq}
\def\be{\ba}
\def\ee{\ea}
\def\stackreb#1#2{\mathrel{\mathop{#2}\limits_{#1}}}
\newcommand{\gsim}{\lower.7ex\hbox{$
\;\stackrel{\textstyle>}{\sim}\;$}}
\newcommand{\lsim}{\lower.7ex\hbox{$
\;\stackrel{\textstyle<}{\sim}\;$}}
\newcommand{\nfour}{${\mathcal N}=4$ }
\newcommand{\ntwo}{${\mathcal N}=2$ }
\newcommand{\ntwon}{${\mathcal N}=2$}
\newcommand{\ntwot}{${\mathcal N}= \left(2,2\right) $ }
\newcommand{\ntwoo}{${\mathcal N}= \left(0,2\right) $ }
\newcommand{\none}{${\mathcal N}=1$ }
\newcommand{\nonen}{${\mathcal N}=1$}
\newcommand{\vp}{\varphi}
\newcommand{\pt}{\partial}
\newcommand{\ve}{\varepsilon}
\newcommand{\gs}{g^{2}}
\newcommand{\qt}{\tilde q}

%%
\newcommand{\p}{\partial}
\newcommand{\wt}{\widetilde}
\newcommand{\ov}{\overline}
\newcommand{\mc}[1]{\mathcal{#1}}
\newcommand{\md}{\mathcal{D}}

\newcommand{\GeV}{{\rm GeV}}
\newcommand{\eV}{{\rm eV}}
\newcommand{\Heff}{{\mathcal{H}_{\rm eff}}}
\newcommand{\Leff}{{\mathcal{L}_{\rm eff}}}
\newcommand{\el}{{\rm EM}}
\newcommand{\uflavor}{\mathbf{1}_{\rm flavor}}
\newcommand{\lgr}{\left\lgroup}
\newcommand{\rgr}{\right\rgroup}

\newcommand{\Mpl}{M_{\rm Pl}}
\newcommand{\suc}{{{\rm SU}_{\rm C}(3)}}
\newcommand{\sul}{{{\rm SU}_{\rm L}(2)}}
\newcommand{\sutw}{{\rm SU}(2)}
\newcommand{\suth}{{\rm SU}(3)}
\newcommand{\ue}{{\rm U}(1)}
%%%%%%%%%%%%%%%%%%%%%%%%%%%%%%%%%%%%%%%
%  Slash character...
\def\slashed#1{\setbox0=\hbox{$#1$}             % set a box for #1
   \dimen0=\wd0                                 % and get its size
   \setbox1=\hbox{/} \dimen1=\wd1               % get size of /
   \ifdim\dimen0>\dimen1                        % #1 is bigger
      \rlap{\hbox to \dimen0{\hfil/\hfil}}      % so center / in box
      #1                                        % and print #1
   \else                                        % / is bigger
      \rlap{\hbox to \dimen1{\hfil$#1$\hfil}}   % so center #1
      /                                         % and print /
   \fi}                                        %

%%EXAMPLE:  $\slashed{E}$ or $\slashed{E}_{t}$

%%

\newcommand{\LN}{\Lambda_\text{SU($N$)}}
\newcommand{\sunu}{{\rm SU($N$) $\times$ U(1)} }
\newcommand{\sunun}{{\rm SU($N$) $\times$ U(1)}}
\def\cfl {$\text{SU($N$)}_{\rm C+F}$ }
\newcommand{\mUp}{m_{\rm U(1)}^{+}}
\newcommand{\mUm}{m_{\rm U(1)}^{-}}
\newcommand{\mNp}{m_\text{SU($N$)}^{+}}
\newcommand{\mNm}{m_\text{SU($N$)}^{-}}
\newcommand{\AU}{\mc{A}^{\rm U(1)}}
\newcommand{\AN}{\mc{A}^\text{SU($N$)}}
\newcommand{\aU}{a^{\rm U(1)}}
\newcommand{\aN}{a^\text{SU($N$)}}
\newcommand{\baU}{\ov{a}{}^{\rm U(1)}}
\newcommand{\baN}{\ov{a}{}^\text{SU($N$)}}
\newcommand{\lU}{\lambda^{\rm U(1)}}
\newcommand{\lN}{\lambda^\text{SU($N$)}}
%\newcommand{\Tr}{{\rm Tr\,}}
\newcommand{\bxir}{\ov{\xi}{}_R}
\newcommand{\bxil}{\ov{\xi}{}_L}
\newcommand{\xir}{\xi_R}
\newcommand{\xil}{\xi_L}
\newcommand{\bzl}{\ov{\zeta}{}_L}
\newcommand{\bzr}{\ov{\zeta}{}_R}
\newcommand{\zr}{\zeta_R}
\newcommand{\zl}{\zeta_L}
\newcommand{\nbar}{\ov{n}}
\newcommand{\nnbar}{n\ov{n}}

\newcommand{\loU}{\lambda_0^{\rm U(1)}}
\newcommand{\llU}{\lambda_1^{\rm U(1)}}
\newcommand{\loN}{\lambda_0^\text{SU($N$)}}
\newcommand{\llN}{\lambda_1^\text{SU($N$)}}
\newcommand{\poU}{\psi_0^{\rm U(1)}}
\newcommand{\plU}{\psi_1^{\rm U(1)}}
\newcommand{\poN}{\psi_0^\text{SU($N$)}}
\newcommand{\plN}{\psi_1^\text{SU($N$)}}

\newcommand{\CPC}{CP($N-1$)$\times$C }
\newcommand{\CPCn}{CP($N-1$)$\times$C}

\newcommand{\MN}{M_\text{SU($N$)}}
\newcommand{\MU}{M_{\rm U(1)}}

\newcommand{\Tr}{\text{Tr}}
\newcommand{\Ts}{\text{Ts}}
\newcommand{\dm}{\hat{{\scriptstyle \Delta} m}}
\newcommand{\dmdag}{\hat{{\scriptstyle \Delta} m}{}^\dag}

\newcommand{\ie}{{\it i.e.} }

\begin{document}

	During discussion of the problem of strings at large $ \mu $ with Alexei, we discovered the following.
	In the undeformed \ntwo bulk theory with quark masses we found the generalization of the
	ansatz for the non-Abelian adjoint field $ a^a $ known in SU(2) theory
\beq
\label{su(2)}
	a^a\, \frac{\tau^a}{2}    ~~=~~    -\, \frac{\Delta m}{2\sqrt{2}}\, 
		\lgr  \tau^3\, \omega(r)  ~+~  S^3\, S^a\,\tau^a\, (1 \,-\, \omega(r))  \rgr,
\eeq
	where $ \omega(r) $ is the adjoint profile function 
\beq
\label{omega}
	\omega(r)    ~~=~~    \frac{\phi_1(r)}{\phi_2(r)}\,,
	\qquad\qquad
	\omega(0)    ~~=~~    0\,,
	\qquad
	\omega(\infty)    ~~=~~    1\,,
\eeq
	to the case of the theory with SU(N) gauge group
\beq
\label{ansatz}
	\aN    ~~=~~    a^a\, T^a    ~~=~~
	-\, \frac{1}{\sqrt{2}}\, 
	\lgr \dm  ~-~  (1 \,-\, \omega(r)) \Big[\, \nnbar\, \big[\, \nnbar,\, \dm \,\big] \,\Big] \rgr.
\eeq
	Here $ \dm $ is a diagonal matrix of mass differences
\beq
	\dm{}_j    ~~=~~    m_j  ~-~  m\,,
	\qquad\qquad
	m    ~~=~~    \frac{1}{N}\,\sum\, m_j\,,
\eeq
	and $ \omega(r) $ is the same function.
	The average mass $ m $ enters only the U(1) ``adjoint'', which does not develop a profile
\beq
	\aU    ~~=~~    \langle \aU \rangle    ~~=~~  -\,\frac{m}{\sqrt{2}}\,.
\eeq

	The ansatz \eqref{ansatz} has the following properties,
\begin{itemize}
\item[$\cdot$]
	it is a traceless matrix

\item[$\cdot$]
	at $ r ~\longrightarrow~ \infty $ it approaches the vacuum value 
\beq
	\aN    ~~\overset{r\,\to\,\infty}{\longrightarrow}~~    
	\langle \aN \rangle    \;~~=~~\;    -\, \frac{\dm}{\sqrt{2}}
\eeq
	required by the potential


\item[$\cdot$]    
	at the string core it turns into a matrix which commutes with the gauge field (\ie effectively with $ \nnbar $)
\[
	\big[~ \nnbar ~,~ \aN(0) ~\big]    ~~=~~    0\,,
\]
	which is needed for finiteness of the string tension


\item[$\cdot$]  
	when $ \vec{n} $ is taken to be in any vacuum $ \vec{n}_\text{vac} ~=~  (\, 0,~ ...,~ 1,~ ...,~ 0 \,) $
	the ansatz again takes its vacuum value
\beq
\label{aNvac}
	\aN \big(\, \vec{n}_\text{vac} \,\big)    ~~=~~    -\, \frac{\dm}{\sqrt{2}}
\eeq


\item[$\cdot$]    
	it turns into the result \eqref{su(2)} for SU(2) when $ N \,=\, 2 $, \ie it is a correct generalization.

\end{itemize}

	Equation \eqref{ansatz} is not the only generalization of the SU(2) formula \eqref{su(2)}.
	In fact, there is a ``direct'' generalization of the SU(2) case via
\begin{align}
%
\notag
	\frac{\Delta m}{2}\, \tau^3    & ~~\longrightarrow~~    \dm\,,\\[2mm]
%
	\frac{S^a\, \tau^a}{2}    & ~~\longrightarrow~~    \nnbar \,-\, 1/N \,,\\[3mm]
%
\notag
	S^3    & ~~\longrightarrow~~     (\nbar\, \tau^3\, n)\,.
\end{align}
	This however would give
\[
	-\, \frac{1}{\sqrt{2}}\, \lgr  \dm \cdot \omega(r)  \;~+~\;  
	2\, ( 1 - \omega(r) ) \cdot (\nbar\, \dm\, n) \, \big( \nnbar \,-\, 1/N \big) \rgr.
\]
	This expression, although correctly reducing to the SU(2) case, fails on several points.
	In particular, upon substution into the original Lagrangian it delivers just rubbish.
	Also it is easy to see that this expression does not satisfy Eq.~\eqref{aNvac} ---
	when $ \vec{n} \,=\, \vec{n}_\text{vac} $, the variable part (proportional to $ \omega(r) $) does not go away ---
	and so cannot be a good physical generalization.
	It is a peculiarity of SU(2) case only that this expression and the ansatz \eqref{ansatz} match, and 
	either is physical.

	The substitution of the ansatz \eqref{ansatz} into the Lagrangian gives
\begin{align}
%
\notag
	&
	\qquad
	\mc{L}    ~~\supset~~  \\
%
\notag
	&
	\frac{4\pi}{g_2^2}\, \int\, r\, dr\,
		\Biggl\lgroup  (\p_r\, \omega)^2 \,+\, \frac{1}{r^2}\, f_N^2\, \omega^2 
			~+~  g_2^2 \lgr \omega\, \big( \phi_1 \,-\, \phi_2 \big)^2  ~+~
				\frac{1}{2}\, (1 - \omega)^2\, \big(\, \phi_1^2 \,+\, \phi_2^2 \,) \rgr
		\Biggr\rgroup
	\\
%
\label{mpot}
	&
	\qquad~~~~\,
	\times
	\lgr  \big( \nbar\, \big| \dm \big|^2\, n \big)  ~-~  \big|\, ( \nbar\, \dm\, n ) \,\big|^2  \rgr
	~~+~~
	O \big( \dm^4 \big).
\end{align}
	The corrections $ O \big( \dm^4 \big) $ arise from the commutator $ \big[\, \baN \,,\, \aN \big] $ 
	in the $D$-term of the potential and are suppressed by the inverse width of the string 
\[
	O \big( \dm^4 \big)    ~~\propto~~   -\, \frac{2\pi}{4}\, \int\, r\, dr\, \big( 1 \,-\, \omega^2 \big)^2
	\,\cdot\, \big| \dm \big|^4     ~~\sim~~    \big| \dm \big|^2 \cdot \frac{ \big| \dm \big|^2 }{ \xi }\,,
\]
	as compared to the potential \eqref{mpot}, for which we know that its transverse integral equals one 
	in the supersymmetric case. 
	Since we are not tracking anything beyond $ O \big( \dm^2 \big) $, these corrections are incomplete
	and therefore would have cancelled had we carefully included all such terms --- 
	we know that in the exact answer these corrections are absent.
	We add that in SU(2) case such corrections do not even appear in Eq.~\eqref{mpot} 
	as $ a^\text{SU(2)} $ and $ \ov{a}{}^\text{SU(2)} $ commute.
	Summarizing, we observe that the ansatz \eqref{ansatz} correctly reproduces the twisted--mass potential 
	on the string, 
\beq
	V    ~~=~~    \big( \nbar\, \big| \dm \big|^2\, n \big)  ~~-~~  \big|\, ( \nbar\, \dm\, n ) \,\big|^2\,,
\eeq
	with the correct normalizing integral, up to terms of order $ \Delta m^4 $.
	

\vspace{0.8cm}
\centerline{*\qquad\qquad\qquad*\qquad\qquad\qquad*}
\vspace{0.8cm}


	As another application of the ansatz \eqref{ansatz}, we compute and confirm the linear correction 
	to the world-sheet potential due to supersymmetry breaking parameters $ \mu_1 $ and $ \mu_2 $.

	The relevant terms in the bulk potential are the following $ F $-terms,
\beq
\label{Fterms}
	\frac{1}{2}\,g_1^2~ \Big|\, \Tr~ q\, \wt{q} ~+~ 2\, \sqrt{N}\, \mu_1 \cdot \aU \Big|^2
	~~+~~
	g_2^2~ \Tr\, \Big|\, \Ts~ q\, \wt{q} ~+~ \sqrt{2}\, \mu_2 \cdot \aN \,\Big|^2\,.
\eeq
	Here ``$ \text{Ts} $'' denotes the traceless part of a matrix.
	We expand this potential to the linear power in $ \dm $, using the zero-order (in $ \dm $) 
	quark profiles $ q $ and the first-order ansatz \eqref{ansatz} for $ \aN $.

	The first term does not contain $ \dm $ in this approximation and therefore is just
	part of the average (\ie zero-order) string tension
\beq
\label{avtension}
	2 \pi\, \big|\, \hat{\xi} \,\big|    ~~=~~    2 \pi \cdot \Big| 2\, \sqrt{2/N}\, \mu_1\, m \Big| \,.
\eeq

	The second term does produce the expected linear corrections.
	We notice that the commutator part (the second term) of the adjoint field 
\beq
	\aN    ~~=~~    
	-\, \frac{1}{\sqrt{2}}\, 
	\lgr \dm  ~-~  (1 \,-\, \omega(r)) \Big[\, \nnbar\, \big[\, \nnbar,\, \dm \,\big] \,\Big] \rgr.
\eeq
	does not contribute at the linear order since
\[
	\Tr~\, \nnbar \, \big[\, \nnbar \,,\, * \,\big]    ~~=~~    0\,.
\]
	This way, only the vacuum value $ -\, \dm/\sqrt{2} $ of $ \aN $ plays the r\^ole here.
	The linear terms in the string potential are found to be,
\beq
	2\pi \cdot 
	\lgr
		\mu_2\, \big(\nbar\, \dm\, n\big) \cdot \frac{\ov{\mu_1\, m}}{\big|\, \mu_1\, m \,|}
		~+~
		\ov{\mu}{}_2\, \big(\nbar\, \dmdag\, n\big) \cdot \frac{\mu_1\, m}{\big|\, \mu_1\, m \,|}
	\rgr.
\eeq
	Together with Eq.~\eqref{avtension}, they constitute the linear expansion of 
\begin{align}
%
\notag
	&
	2 \pi \cdot \big|\, \hat{\xi}  ~+~  2\, \mu_2\, \big(\nbar\, \dm\, n\big) \,\big|    ~~=~~  
	\\
%
\notag
	&
	\qquad\qquad\quad
	~~=~~
	2\pi \cdot
	\lgr
		\big|\, \hat{\xi} \,\big|
		~+~
		\mu_2\, \big(\nbar\, \dm\, n\big) \cdot \frac{\ov{\hat{\xi}}}{\big|\, \hat{\xi} \,\big|}
		~+~
		\ov{\mu}{}_2\, \big(\nbar\, \dmdag\, n\big) \cdot \frac{\hat{\xi}}{\big|\, \hat{\xi} \,\big|}
		~+~
		\dots
	\rgr,
\end{align}
	which gives the right string tensions in the vacua $ \vec{n}{}_\text{vac} $.

%%%%%%%%%%%%%%%%%%%%%%%%%%%%%%%%%%%%%%%%%%%%%%%%%%%%%%%%%%%%%%%%%%%%%%%%%%%%%%%%%%%%%%%%%%%%%%%%%%%%%%%%%%%%%%%%%%%%%%%%%%%%%%%%%%%%%%%%%%%%%%%%%%%%%%%%%%%%%%%%

\vspace{0.8cm}
\centerline{*\qquad\qquad\qquad*\qquad\qquad\qquad*}
\vspace{0.8cm}


	If you agree with this, I will send you a draft promptly

\end{document}
